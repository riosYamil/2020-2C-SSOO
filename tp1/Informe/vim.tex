\begin{section}{Vim}

\begin{quoting}
Vim es uno de los editores de texto que vienen por defecto instalados en todo sistema Linux.\\
Tipee “man vim” para investigar un poco sus características.\end{quoting}
\begin{lstlisting}[style=Ubuntu]
~/uade_workplace/ssoo/tp1
grios@personal-pc:~$ man vim

VIM(1)                   General Commands Manual                  VIM(1)

NAME
	...
SYNOPSIS
       vim [options] [file ..]

DESCRIPTION
       Vim is a text editor that is upwards compatible to Vi.  It can be
       used to edit all kinds of plain text.  It  is  especially  useful
       for editing programs.

\end{lstlisting}

\begin{quoting}
Para crear un archivo y editarlo con el vim, tipee “vim archivo.txt”. \\
Para comenzar a escribir debe ingresar al modo edición, presionando la tecla a. Escriba un poco de
texto y luego salga del modo de edición presionando <ESC>.\\
Para grabar el archivo, presione: w (estando fuera del modo edición).\\
Para salir del editor, presione: q (estando fuera del modo edición).\\
Para profundizar sobre el vim (más adelante, cuando sea necesario) puede tipear “vim”. Así entrará a
una pantalla desde la cual podrá tipear “:help” y ingresar a la ayuda. También puede tipear vimtutor, un
programa que ofrece un tutorial completo del vim.
\end{quoting}

\begin{lstlisting}[style=Ubuntu]
~/uade_workplace/ssoo/tp1
grios@personal-pc:~$ vim archivo.txt

\end{lstlisting}

Inicamos VIM
\begin{lstlisting}[style=Ubuntu]

~                                                                
~                                                                
~                                                                
~                                                                
~                                                                
~                                                                
~                                                                
~                                                                
~                                                                
~                                                                
~                                                                
~                                                                
~                                                                
~                                                                
~                                                                
~                                                                
                                               1,0-1         All
\end{lstlisting}

Inicamos modo de escritura con la tecla 'a' e ingresamos la frase 'Hello world'
\begin{lstlisting}[style=Ubuntu]
Hello world                                                               
~                                                                
~                                                                
~                                                                
~                                                                
~                                                                
~                                                                
~                                                                
~                                                                
~                                                                
~                                                                
~                                                                
~                                                                
~                                                                
~                                                                
~                                                                
-- INSERT --                                   1,1           All
\end{lstlisting}

Salimos de modo edición y presionamos 'w' para guardar el comando 
\begin{lstlisting}[style=Ubuntu]
Hello world
~                                                                
~                                                                
~                                                                
~                                                                
~                                                                
~                                                                
~                                                                
~                                                                
~                                                                
~                                                                
~                                                                
~                                                                
~                                                                
~                                                                
~                                                                
~                                                                
:w
\end{lstlisting}

\begin{lstlisting}[style=Ubuntu]
Hello world
~                                                                
~                                                                
~                                                                
~                                                                
~                                                                
~                                                                
~                                                                
~                                                                
~                                                                
~                                                                
~                                                                
~                                                                
~                                                                
~                                                                
~                                                                
~                                                                
"archivo.txt" [New] 1L, 12C written                   1,11          All
\end{lstlisting}

Salimos de vim presionando q.
\begin{lstlisting}[style=Ubuntu]
Hello world
~                                                                
~                                                                
~                                                                
~                                                                
~                                                                
~                                                                
~                                                                
~                                                                
~                                                                
~                                                                
~                                                                
~                                                                
~                                                                
~                                                                
~                                                                
~                                                                
:q
\end{lstlisting}

Usamos el comando 'cat' para ver el contenido de 'archivo.txt'
\begin{lstlisting}[style=Ubuntu]
~/uade_workplace/ssoo/tp1
grios@personal-pc:~$ cat archivo.txt 
Hello world

\end{lstlisting}

En los comandos de arriba se creó un archivo.txt con vim, donde se escribió 'Hello world'. \\

\begin{quoting}
En caso de no sentirse cómodo con el vim, investigue el programa “nano”
\end{quoting}

Tal como vim, nano es un editor de texto cuya curva de aprendizaje pronunciada.

\end{section}