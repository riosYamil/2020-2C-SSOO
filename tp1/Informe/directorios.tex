\begin{section}{Directorios}

\begin{quoting}
¿Para qué se usa el comando cd?
\end{quoting}

Si bien usamos este comando durante casi todo el TP, tratamos de responder esta parte del TP ejecutando el siguiente comando sin exito:

\begin{lstlisting}[style=Ubuntu]
~/uade_workplace/ssoo/tp1
grios@personal-pc:~$ man cd
No manual entry for cd

\end{lstlisting}

En base a nuestra experiencia, podemos decir que el comando cd (change directory) sirve para cambiar de directorios dentro una sesión del terminal\\

\begin{quoting}
Ejecute las siguientes variantes de cd y observe cuál fue el resultado
obtenido:
\end{quoting}

\begin{itemize}
\item textbf{cd /}

\begin{lstlisting}[style=Ubuntu]
~
grios@personal-pc:~$ cd /

/
grios@personal-pc:~$ pwd
/
\end{lstlisting}
Nos lleva al directorio root del sistema.

\item textbf{cd}
\begin{lstlisting}[style=Ubuntu]
/
grios@personal-pc:~$ cd 

~
grios@personal-pc:~$ pwd
/home/grios
\end{lstlisting}
Nos dirige al directorio root del usuario,

\item textbf{cd /etc}
\begin{lstlisting}[style=Ubuntu]
~
grios@personal-pc:~$ cd /etc

/etc
grios@personal-pc:~$ pwd
/etc
\end{lstlisting}
Nos dirige al directorio root llamado '/etc'. En este directorio se persisten archivos de configuración del sistema operativo y paquetes instalados.

\item textbf{cd .}
\begin{lstlisting}[style=Ubuntu]
/etc
grios@personal-pc:~$ cd .

/etc
grios@personal-pc:~$ pwd
/etc
\end{lstlisting}
Nos direcciona al directorio actual.

\item textbf{cd ..}
\begin{lstlisting}[style=Ubuntu]
/etc
grios@personal-pc:~$ cd ..

/
grios@personal-pc:~$ pwd
/
\end{lstlisting}
Nos redireccionó al directorio superior, es decir, directorio root del sistema.
\end{itemize}


\begin{quoting}
Investigue que hacen los comandos:
\end{quoting}

\begin{itemize}
\item textbf{mkdir}
Este comando nos permite crear un directorio vacío.
\item textbf{rmdir}
Este comando nos permite borrar un directorio vacío.
\item textbf{rm}
Este comando nos permite borrar archivos, por defaul no borra directorios pero podría hacerlo si se le setea el flag '-r'
\end{itemize}


\begin{quoting}
Borre un directorio que no se encuentra vacío.
\end{quoting}
\begin{lstlisting}[style=Ubuntu]
~/uade_workplace/ssoo/tp1
grios@personal-pc:~$ mkdir temporal_dir

~/uade_workplace/ssoo/tp1
grios@personal-pc:~$ ls temporal_dir 

~/uade_workplace/ssoo/tp1
grios@personal-pc:~$ touch temporal_dir/temp_file

~/uade_workplace/ssoo/tp1
grios@personal-pc:~$ rm -r temporal_dir
\end{lstlisting}

\begin{quoting}
Borre un directorio que no se encuentra vacío.
\end{quoting}
\begin{lstlisting}[style=Ubuntu]
~/uade_workplace/ssoo/tp1
grios@personal-pc:~$ cd

~
grios@personal-pc:~$ mkdir undir

~
grios@personal-pc:~$ cd undir

\end{lstlisting}


\begin{quoting}
Ingrese a dicho directorio, y tipee lo siguiente para crear muchos archivos “while (true) do ps >
\$RANDOM.text; done;”. Tipee ctrl.+c luego de 5 seg para finalizar el comando. Luego tipee ls para
corroborar la creación de los archivos.
\end{quoting}

Un detalle en este punto, simplemente notamos que el comando solicitado no genera los N archivos esperados, sino que se generaba uno solo. Supusimos que esto es porque la función definida dentro de \$RANDOM devuelve un valor distinto cada vez que es consultado por el comando 'echo \$RANDOM', como en la instrucción propuesta sólo se usa para persistir un file sin llamar al comando 'echo', el valor que devuelve \$RANDOM nunca cambia haciendo que siempre se genere un solo archivo.

Dicho esto, cambiarmos el comando propuesto para que se ejecute un 'echo \$RANDOM' luego de persistido el archivo.

\begin{lstlisting}[style=Ubuntu]
~/undir
grios@personal-pc:~$ while (true) do ps > $RANDOM.txt; echo $RANDOM; done;

~
grios@personal-pc:~$ ls
10054.txt  15581.txt  1981.txt   24335.txt  30462.txt  5262.txt
10422.txt  15784.txt  19847.txt  24355.txt  30603.txt  5319.txt
10623.txt  15834.txt  19848.txt  2452.txt   30675.txt  5322.txt
10624.txt  15995.txt  20074.txt  24673.txt  30903.txt  5489.txt
10635.txt  15997.txt  20197.txt  24884.txt  3101.txt   5541.txt
10805.txt  16047.txt  20308.txt  25109.txt  31260.txt  5675.txt
10829.txt  16049.txt  20348.txt  25166.txt  31407.txt  572.txt
10982.txt  16081.txt  2037.txt   25269.txt  31550.txt  5775.txt
11000.txt  16094.txt  20386.txt  25333.txt  31599.txt  5934.txt
11582.txt  16335.txt  20388.txt  25562.txt  31631.txt  5978.txt
11632.txt  16398.txt  20663.txt  25625.txt  31781.txt  6138.txt
12090.txt  16446.txt  20782.txt  25775.txt  31819.txt  6372.txt
12106.txt  1654.txt   20877.txt  2584.txt   31839.txt  6437.txt
12120.txt  16668.txt  21009.txt  26048.txt  31856.txt  6476.txt
12139.txt  16743.txt  21023.txt  26061.txt  31934.txt  6506.txt
12237.txt  16871.txt  21321.txt  26135.txt  31940.txt  6525.txt
12465.txt  16944.txt  21527.txt  26330.txt  31941.txt  6544.txt
12530.txt  17020.txt  2156.txt   26343.txt  31965.txt  658.txt
12689.txt  17257.txt  21634.txt  2641.txt   32057.txt  6632.txt
12728.txt  17324.txt  21863.txt  26527.txt  32088.txt  6703.txt
12813.txt  17425.txt  21919.txt  26704.txt  32288.txt  6890.txt
12880.txt  17618.txt  21935.txt  26719.txt  32539.txt  6937.txt
12932.txt  17631.txt  21956.txt  26768.txt  32555.txt  7029.txt
13002.txt  17648.txt  21962.txt  2685.txt   32649.txt  7343.txt
13176.txt  17676.txt  22065.txt  26909.txt  32667.txt  7427.txt
13214.txt  1774.txt   2247.txt   27087.txt  32717.txt  7487.txt
13261.txt  17811.txt  22538.txt  27137.txt  32766.txt  7654.txt
13514.txt  18180.txt  22549.txt  27153.txt  3287.txt   7793.txt
13716.txt  18183.txt  22639.txt  27273.txt  3321.txt   8147.txt
13879.txt  1821.txt   22879.txt  27994.txt  
\end{lstlisting}


\begin{quoting}
Tipee “rm *” e investigue que pasó.
\end{quoting}\\

\begin{lstlisting}[style=Ubuntu]
~/uade_workplace/ssoo/tp1
grios@personal-pc:~$ rm *
zsh: sure you want to delete more than 100 files in /home/grios/undir [yn]? y

grios@personal-pc:~$ ls

\end{lstlisting}

El comando rm (remove) borró todos los archivos de la carpeta actual. Esto se dió porque se hizo uso del wild card del asterisco (*), este comodín permite hacer referencia a todos los elemento del directorio actual y es por eso que se borraron.
\end{section}