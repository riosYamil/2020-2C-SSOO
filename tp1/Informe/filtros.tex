\begin{section}{Filtros}

\begin{subsection}{more, less y cat}
\begin{quoting}
¿Cuál es la diferencia de los comandos more, less y cat? Cree un archivo de texto tipeando “ps –fea >
texto” y visualícelo con los distintos comandos.
\end{quoting}\\

\begin{quoting}
Investigue como buscar cadenas de texto cuando se visualiza un archivo con less. ¿Y cómo se hace
para repetir la búsqueda? ¿Y para repetir la búsqueda hacia atrás? Esto le servirá cuando lea páginas del
man! (ya que se leen mediante el less).
\end{quoting}\\

\end{subsection}

\begin{subsection}{tail y head}
\begin{quoting}
¿Cuál es la diferencia entre tail y head?. ¿Qué hace la opción –f del comando tail?
\end{quoting}\\

\begin{quoting}
Loguéese en una Terminal y tipee “echo > a.txt” para crear el archivo “a.txt”. Luego tipee “tail –f
\end{quoting}\\

\begin{quoting}
Desde otra Terminal, tipee “while (true) do date $\gg$ a.txt; sleep 2; done;”.
\end{quoting}\\

\begin{quoting}
Vuelva a la terminal anterior y vea lo que sucede.
\end{quoting}\\

\begin{quoting}
No se olvide de finalizar el comando de la 2da Terminal! (con ctrl.+c)
\end{quoting}\\

\end{subsection}

\begin{subsection}{sort}
\begin{quoting}
¿Qué es lo que realiza el comando sort?
\end{quoting}\\

\end{subsection}

\begin{subsection}{uniq}
\begin{quoting}
¿Qué es lo que realiza el comando uniq?
\end{quoting}\\

\end{subsection}

\begin{subsection}{grep}
\begin{quoting}
¿Para qué sirve?
\end{quoting}\\

\begin{quoting}
Busque en el archivo “texto” todas las líneas que contengan la palabra “root”.
\end{quoting}\\

\end{subsection}

\begin{subsection}{find}
\begin{quoting}
¿Para qué sirve?
\end{quoting}\\

\begin{quoting}
Busque un patrón determinado en un directorio utilizando find y grep.
\end{quoting}\\
\end{subsection}

\end{section}