\begin{section}{Permisos}

\begin{subsection}{ls}
\begin{quoting}
Cree un archivo tipeando “ls > archivo”.
\end{quoting}
\begin{lstlisting}[style=Ubuntu]
~/uade_workplace/ssoo/tp1
grios@personal-pc:~$ ls
enunciado.pdf  Informe

~/uade_workplace/ssoo/tp1
grios@personal-pc:~$ ls > archivo

~/uade_workplace/ssoo/tp1
grios@personal-pc:~$ ls 
archivo enunciado.pdf  Informe
\end{lstlisting}

\begin{quoting}
Tipee ls –l en dicho directorio: los primeros 10 caracteres corresponden a los permisos. Investigue
como se estructuran los permisos de un archivo (puede tipear info y luego ir a la sección de “permisos de
archivo” o “file permissions”).
\end{quoting}
\begin{lstlisting}[style=Ubuntu]
~/uade_workplace/ssoo/tp1
grios@personal-pc:~$ ls -l
total 200
-rw-rw-r-- 1 grios grios     30 ago 23 18:59 archivo
-rw-rw-r-- 1 grios grios 194796 ago 23 02:48 enunciado.pdf
drwxrwxr-x 3 grios grios   4096 ago 23 19:01 Informe
\end{lstlisting}

Los permiso de un archivo se estructuran con 10 caracteres (10 bits), los cuales se ven representados de la siguiente forma:
\\

\begin{bytefield}[endianness=big,bitwidth=5em, bitheight=3ex]{10}
	
	\bitheader{0-9}\\

	\bitbox{1}{-}& 
	\bitbox{1}{r}&
	\bitbox{1}{w}&
	\bitbox{1}{x}
	\bitbox{1}{r}&
	\bitbox{1}{w}&
	\bitbox{1}{x}
	\bitbox{1}{r}&
	\bitbox{1}{w}&
	\bitbox{1}{x}\\

	\bitbox{1}{archivo}& 
	\bitbox{3}{permiso del usuario}&
	\bitbox{3}{permisos de grupo}&
	\bitbox{3}{permisos para otros}
\end{bytefield}\\
Notese que el bit más significativo tiene el caracter '-', el cual indica que corresponde a un archivo.

Donde la primer fila corresponde a permisos básicos que son \textbf{lectura (r), escritura (w) y ejecución (x)}.
Una vez mencinados estos 3 tipos de permisos podemos ver que estos se agrupan de a tercios, donde cada uno corresponde a \textbf{permisos de usuarios, grupos y otros}. 
\end{subsection}


\begin{subsection}{chmod/chown}
\begin{quoting}
Investigue que hacen esos comandos.
\end{quoting}
\begin{itemize}
\item chmod
El comando 'change mode' permite cambiar los permisos de acceso sobre archivos y directorios. 
\item chown
El comando 'change owner' permite el propiertario de archivos o directorios.  
\end{itemize}

\begin{quoting}
Haga que el archivo “archivo” creado anteriormente pueda ser modificado por cualquier usuario.
\end{quoting}
\begin{lstlisting}[style=Ubuntu]
~/uade_workplace/ssoo/tp1
grios@personal-pc:~$ chmod o+rw archivo

~/uade_workplace/ssoo/tp1
grios@personal-pc:~$ ls -l             
total 200
-rw-rw-rw- 1 grios grios     30 ago 23 18:59 archivo
-rw-rw-r-- 1 grios grios 194796 ago 23 02:48 enunciado.pdf
drwxrwxr-x 3 grios grios   4096 ago 23 19:58 Informe
\end{lstlisting}

\begin{quoting}
Compruebe que logró el punto anterior logueandose en otra Terminal con otro usuario y
modificando dicho archivo (tipeando nuevamente “ls > archivo”).
\end{quoting}
\begin{lstlisting}[style=Ubuntu]
~/uade_workplace/ssoo/tp1
grios@personal-pc:~$ sudo useradd -m -d $HOME test_user

~/uade_workplace/ssoo/tp1
grios@personal-pc:~$ sudo login test_user
Password: 
Welcome to Ubuntu 20.04 LTS (GNU/Linux 5.4.0-42-generic x86_64)...

$ cd uade_workplace/ssoo/tp1
$ ls 
archivo  enunciado.pdf	Informe
$ ls > archivo
$ ls -l 
total 200
-rw-rw-rw- 1 grios grios     30 ago 23 20:44 archivo
-rw-rw-r-- 1 grios grios 194796 ago 23 02:48 enunciado.pdf
drwxrwxr-x 3 grios grios   4096 ago 23 19:58 Informe

\end{lstlisting}

\begin{quoting}
Loguéese con el usuario original y quite los todos los permisos del archivo (lectura, escritura y
ejecución) a todos los usuarios distintos del dueño y de los que pertenecen al mismo grupo. Luego, haga que
el nuevo dueño del archivo sea el otro usuario.
\end{quoting}
\begin{lstlisting}[style=Ubuntu]
~/uade_workplace/ssoo/tp1
grios@personal-pc:~$ chmod o-rwx,g-rwx archivo

~/uade_workplace/ssoo/tp1
grios@personal-pc:~$ ls -l                     
total 200
-rw------- 1 grios grios     30 ago 23 20:53 archivo
-rw-rw-r-- 1 grios grios 194796 ago 23 02:48 enunciado.pdf
drwxrwxr-x 3 grios grios   4096 ago 23 20:58 Informe


~/uade_workplace/ssoo/tp1
grios@personal-pc:~$ chown test_user archivo
chown: changing ownership of 'archivo': Operation not permitted

~/uade_workplace/ssoo/tp1
grios@personal-pc:~$ sudo chown test_user archivo

~/uade_workplace/ssoo/tp1
grios@personal-pc:~$ ls -l
total 200
-rw------- 1 test_user grios     30 ago 23 20:53 archivo
-rw-rw-r-- 1 grios     grios 194796 ago 23 02:48 enunciado.pdf
drwxrwxr-x 3 grios     grios   4096 ago 23 20:58 Informe

\end{lstlisting}

\begin{quoting}
¿Cómo haría para volver a poseer dicho archivo sin loguearse con el nuevo dueño del archivo?
\end{quoting}
\\
La forma de recuperar la propiedad de un archivo es ejecutando ejecutan el comando chown con permisos de superusuario, por ejemplo:

\begin{lstlisting}[style=Ubuntu]
~/uade_workplace/ssoo/tp1
grios@personal-pc:~$ ls -l
total 200
-rw------- 1 test_user grios     30 ago 23 20:53 archivo
-rw-rw-r-- 1 grios     grios 194796 ago 23 02:48 enunciado.pdf
drwxrwxr-x 3 grios     grios   4096 ago 23 20:58 Informe

~/uade_workplace/ssoo/tp1
grios@personal-pc:~$ sudo chown grios archivo

~/uade_workplace/ssoo/tp1
grios@personal-pc:~$ ls -l
total 204
-rw------- 1 grios grios     30 ago 23 20:53 archivo
-rw-rw-r-- 1 grios grios 194796 ago 23 02:48 enunciado.pdf
drwxrwxr-x 3 grios grios   4096 ago 23 20:58 Informe

\end{lstlisting}

\begin{quoting}
Investigue qué es el “SUID bit” (busque en “man chmod”).
\end{quoting}\\
El SUID bit (setup ID bit) se indica que todo aquel que ejecute el archivo tendrá, durante la ejecución, los mismo privilegios que el usuario owner del archivo.\\

\begin{quoting}
Investigue como aplica la estructura de los permisos a los directorios.
\end{quoting}\\
Al igual que los archivos, los permiso de un directorio se estructuran con 10 caracteres (10 bits), los cuales se ven representados de la siguiente forma:\\
\begin{bytefield}[endianness=big,bitwidth=5em, bitheight=3ex]{10}
	
	\bitheader{0-9}\\

	\bitbox{1}{d}& 
	\bitbox{1}{r}&
	\bitbox{1}{w}&
	\bitbox{1}{x}
	\bitbox{1}{r}&
	\bitbox{1}{w}&
	\bitbox{1}{x}
	\bitbox{1}{r}&
	\bitbox{1}{w}&
	\bitbox{1}{x}\\

	\bitbox{1}{directorio}& 
	\bitbox{3}{permiso del usuario}&
	\bitbox{3}{permisos de grupo}&
	\bitbox{3}{permisos para otros}
\end{bytefield}\\
Notese que el bit más significativo tiene el caracter 'd', el cual indica que corresponde a un directorio.\\

\begin{quoting}
Loguéese como root en otra Terminal y cree un directorio tipeando “mkdir /undir”.
\end{quoting}
\begin{lstlisting}[style=Ubuntu]
~/uade_workplace/ssoo/tp1
grios@personal-pc:~$ su
Password: 

~/uade_workplace/ssoo/tp1
grios@personal-pc:~$ ls -l
total 204
-rw------- 1 grios grios     30 ago 23 20:53 archivo
-rw-rw-r-- 1 grios grios 194796 ago 23 02:48 enunciado.pdf
drwxrwxr-x 3 grios grios   4096 ago 23 20:58 Informe

root@grios-pc:/home/grios/uade_workplace/ssoo/tp1# mkdir undir

root@grios-pc:/home/grios/uade_workplace/ssoo/tp1# ls -l
total 200
-rw-rw-r-- 1 grios grios 194796 ago 23 02:48 enunciado.pdf
drwxrwxr-x 3 grios grios   4096 ago 23 22:16 Informe
drwxr-xr-x 2 root  root    4096 ago 23 22:26 undir
\end{lstlisting}

\begin{quoting}
Haga que cualquier usuario tenga todo tipo de permisos sobre ese directorio.
\end{quoting}
\begin{lstlisting}[style=Ubuntu]
root@grios-pc:/home/grios/uade_workplace/ssoo/tp1# chmod o+rwx undir/

root@grios-pc:/home/grios/uade_workplace/ssoo/tp1# ls -l
total 200
-rw-rw-r-- 1 grios grios 194796 ago 23 02:48 enunciado.pdf
drwxrwxr-x 3 grios grios   4096 ago 23 22:16 Informe
drwxr-xrwx 2 root  root    4096 ago 23 22:26 undir
\end{lstlisting}

\begin{quoting}
Deshaga lo que acaba de hacer, y cree el subdirectorio “subdir” dentro de “/undir”
\end{quoting}
\begin{lstlisting}[style=Ubuntu]
root@grios-pc:/home/grios/uade_workplace/ssoo/tp1# exit
exit

~/uade_workplace/ssoo/tp1
grios@personal-pc:~$ mkdir undir/subdir

~/uade_workplace/ssoo/tp1
grios@personal-pc:~$ ls -l undir 
total 4
drwxrwxr-x 2 grios grios 4096 ago 23 22:33 subdir
\end{lstlisting}

\begin{quoting}
Investigue como cambiar los permisos de manera recursiva sobre /undir para que todos sus
archivos, subdirectorios y archivos dentro de los subdirectorios se vean afectados.
\end{quoting}\\
Para cambiar permisos sobre un directorio de manera recursiva se debe ejecutar 'chmod' con el flag '-R'.

\begin{lstlisting}[style=Ubuntu]
~/uade_workplace/ssoo/tp1
grios@personal-pc:~$ ls -l
total 200
-rw-rw-r-- 1 grios grios 194796 ago 23 02:48 enunciado.pdf
drwxrwxr-x 3 grios grios   4096 ago 23 22:37 Informe
drwxr-xrwx 3 root  root    4096 ago 23 22:33 undir

~/uade_workplace/ssoo/tp1
grios@personal-pc:~$ ls -l undir
total 4
drwxrwxr-x 2 grios grios 4096 ago 23 22:33 subdir

~/uade_workplace/ssoo/tp1
grios@personal-pc:~$ chmod -R o+wrx undir
chmod: changing permissions of 'undir': Operation not permitted

grios@personal-pc:~$ ls -l
total 200
-rw-rw-r-- 1 grios grios 194796 ago 23 02:48 enunciado.pdf
drwxrwxr-x 3 grios grios   4096 ago 23 22:37 Informe
drwxr-xrwx 3 root  root    4096 ago 23 22:33 undir

~/uade_workplace/ssoo/tp1
grios@personal-pc:~$ ls -l undir
total 4
drwxrwxrwx 2 grios grios 4096 ago 23 22:33 subdir

\end{lstlisting}

\end{subsection}

\end{section}