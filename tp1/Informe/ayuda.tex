\begin{section}{Ayuda}

\begin{subsection}{Comando man}

\begin{quoting}
man: es un programa que formatea y muestra las páginas del manual de referencia del sistema. El
formato de uso básico es “man tema” donde tema es el nombre de la página del manual que se quiere ver.
\end{quoting}

\begin{quoting}
¿Qué tipo de información provee el man, como la organiza internamente y como busca dentro de la misma? Para saberlo, tipee “man man” (sin comillas), use los cursores UP y DOWN para recorrer la pagina.
\end{quoting}
El comando man nos provee un manual de referencia de cualquier comando que querramos consultar. 

La forma en que estructura la información es mediante secciones, las cuales pueden ser:
\begin{itemize}
\item NAME
\item SYNOPSIS
\item CONFIGURATION 
\item DESCRIPTION 
\item OPTIONS
\item EXIT STATUS 
\item RETURN VALUE
\item ERRORS 
\item ENVIRONMENT
\item FILES
\item VERSIONS
\item CONFORMING TO
\item NOTES
\item BUGS
\item EXAMPLE 
\item AUTHORS
\item SEE ALSO.
\end{itemize}

Cuando no sabemos como se llama el comando que queremos buscar, se pude hacer un busqueda por palabra. 'man -k your\_favorite\_word' el cual nos va a mostrar muchos posibles comando que sean de nusetro interes. 

\begin{lstlisting}[style=Ubuntu]
~/uade_workplace/ssoo/tp1
grios@personal-pc:~$ man -k shutdown

shutdown (2)         - shut down part of a full-duplex connection
shutdown (8)         - Halt, power-off or reboot the machine
systemd-backlight (8) - Load and save the display backlight brightness at boot and shutdown
systemd-backlight@.service (8) - Load and save the display backlight brightness at boot and shutdown
systemd-halt.service (8) - System shutdown logic
systemd-kexec.service (8) - System shutdown logic
systemd-poweroff.service (8) - System shutdown logic
systemd-random-seed (8) - Load and save the system random seed at boot and shutdown
systemd-random-seed.service (8) - Load and save the system random seed at boot and shutdown
systemd-reboot.service (8) - System shutdown logic
systemd-shutdown (8) - System shutdown logic
systemd-update-utmp (8) - Write audit and utmp updates at bootup, runlevel changes and shutdown
systemd-update-utmp-runlevel.service (8) - Write audit and utmp updates at bootup, runlevel changes and shutdown
systemd-update-utmp.service (8) - Write audit and utmp updates at bootup, runlevel changes and shutdown
systemd-user-sessions (8) - Permit user logins after boot, prohibit user logins at shutdown
systemd-user-sessions.service (8) - Permit user logins after boot, prohibit user logins at shutdown


\end{lstlisting}

\begin{quoting}
Investigue que hace el comando ls tipeando “man ls”.
\end{quoting}

\begin{lstlisting}[style=Ubuntu]
~/uade_workplace/ssoo/tp1
grios@personal-pc:~$ man ls

LS(1)                                                              User Commands                                                             LS(1)

NAME
       ls - list directory contents

SYNOPSIS
       ls [OPTION]... [FILE]...

DESCRIPTION
       List information about the FILEs (the current directory by default).  Sort entries alphabetically if none of -cftuvSUX nor --sort is specified.

       Mandatory arguments to long options are mandatory for short options too.

       -a, --all
              do not ignore entries starting with .

       -A, --almost-all
              do not list implied . and ..

       --author
              with -l, print the author of each file

       -b, --escape
              print C-style escapes for nongraphic characters

       --block-size=SIZE
              with -l, scale sizes by SIZE when printing them; e.g., '--block-size=M'; see SIZE format below

       -B, --ignore-backups
              do not list implied entries ending with ~

       -c     with -lt: sort by, and show, ctime (time of last modification of file status information); with -l: show ctime  and  sort  by  name;
              otherwise: sort by ctime, newest first

       -C     list entries by columns

       --color[=WHEN]
              colorize the output; WHEN can be 'always' (default if omitted), 'auto', or 'never'; more info below

       -d, --directory
              list directories themselves, not their contents

       -D, --dired
              generate output designed for Emacs' dired mode

       -f     do not sort, enable -aU, disable -ls --color

       -F, --classify
              append indicator (one of */=>@|) to entries

       --file-type
              likewise, except do not append '*'

       --format=WORD
              across -x, commas -m, horizontal -x, long -l, single-column -1, verbose -l, vertical -C

       --full-time
              like -l --time-style=full-iso

       -g     like -l, but do not list owner

       --group-directories-first
              group directories before files;

              can be augmented with a --sort option, but any use of --sort=none (-U) disables grouping

       -G, --no-group
              in a long listing, don't print group names

       -h, --human-readable
              with -l and -s, print sizes like 1K 234M 2G etc.

       --si   likewise, but use powers of 1000 not 1024

       -H, --dereference-command-line
              follow symbolic links listed on the command line

       --dereference-command-line-symlink-to-dir
              follow each command line symbolic link

              that points to a directory

       --hide=PATTERN
              do not list implied entries matching shell PATTERN (overridden by -a or -A)

       --hyperlink[=WHEN]
              hyperlink file names; WHEN can be 'always' (default if omitted), 'auto', or 'never'

       --indicator-style=WORD
              append indicator with style WORD to entry names: none (default), slash (-p), file-type (--file-type), classify (-F)

       -i, --inode
              print the index number of each file

       -I, --ignore=PATTERN
              do not list implied entries matching shell PATTERN

       -k, --kibibytes
              default to 1024-byte blocks for disk usage; used only with -s and per directory totals

       -l     use a long listing format

       -L, --dereference
              when showing file information for a symbolic link, show information for the file the link references rather than for the link itself

       -m     fill width with a comma separated list of entries

       -n, --numeric-uid-gid
              like -l, but list numeric user and group IDs

       -N, --literal
              print entry names without quoting

       -o     like -l, but do not list group information

       -p, --indicator-style=slash
              append / indicator to directories

       -q, --hide-control-chars
              print ? instead of nongraphic characters

       --show-control-chars
              show nongraphic characters as-is (the default, unless program is 'ls' and output is a terminal)

       -Q, --quote-name
              enclose entry names in double quotes

       --quoting-style=WORD
              use  quoting  style  WORD for entry names: literal, locale, shell, shell-always, shell-escape, shell-escape-always, c, escape (over rides QUOTING_STYLE environment variable) 

       -r, --reverse
              reverse order while sorting

       -R, --recursive
              list subdirectories recursively

       -s, --size
              print the allocated size of each file, in blocks

       -S     sort by file size, largest first

       --sort=WORD
              sort by WORD instead of name: none (-U), size (-S), time (-t), version (-v), extension (-X)

       --time=WORD
              with -l, show time as WORD instead of default modification time: atime or access or use (-u); ctime or status (-c); also use  specified time as sort key if --sort=time (newest first)

       --time-style=TIME_STYLE
              time/date format with -l; see TIME_STYLE below

       -t     sort by modification time, newest first

       -T, --tabsize=COLS
              assume tab stops at each COLS instead of 8

       -u     with -lt: sort by, and show, access time; with -l: show access time and sort by name; otherwise: sort by access time, newest first

       -U     do not sort; list entries in directory order

       -v     natural sort of (version) numbers within text

       -w, --width=COLS
              set output width to COLS.  0 means no limit

       -x     list entries by lines instead of by columns

       -X     sort alphabetically by entry extension

       -Z, --context
              print any security context of each file

       -1     list one file per line.  Avoid '\n' with -q or -b

       --help display this help and exit

       --version
              output version information and exit

       The SIZE argument is an integer and optional unit (example: 10K is 10*1024).  Units are K,M,G,T,P,E,Z,Y (powers of 1024) or KB,MB,... (powers of 1000).

       The TIME_STYLE argument can be full-iso, long-iso, iso, locale, or +FORMAT.  FORMAT is interpreted like in  date(1).   If  FORMAT  is  FORMAT1<newline>FORMAT2, then FORMAT1 applies to non-recent files and FORMAT2 to recent files.  TIME_STYLE prefixed with 'posix-' takes effect
       only outside the POSIX locale.  Also the TIME_STYLE environment variable sets the default style to use.

       Using color to distinguish file types is disabled both by default and with --color=never.  With --color=auto, ls  emits  color  codes  only
       when standard output is connected to a terminal.  The LS_COLORS environment variable can change the settings.  Use the dircolors command to
       set it.

   Exit status:
       0      if OK,

       1      if minor problems (e.g., cannot access subdirectory),

       2      if serious trouble (e.g., cannot access command-line argument).

AUTHOR
       Written by Richard M. Stallman and David MacKenzie.

REPORTING BUGS
       GNU coreutils online help: <https://www.gnu.org/software/coreutils/>
       Report ls translation bugs to <https://translationproject.org/team/>

COPYRIGHT
       Copyright  2018 Free Software Foundation, Inc.  License GPLv3+: GNU GPL version 3 or later <https://gnu.org/licenses/gpl.html>.
       This is free software: you are free to change and redistribute it.  There is NO WARRANTY, to the extent permitted by law.

SEE ALSO
       Full documentation at: <https://www.gnu.org/software/coreutils/ls>
       or available locally via: info '(coreutils) ls invocation'

GNU coreutils 8.30                                                September 2019                                                             LS(1)

\end{lstlisting}
El comando muestra qué un detalle amplio de lo que se peude hacer con el comando 'ls'.\\

\begin{quoting}
Liste de manera ordenada (por tamaño del archivo) mediante el comando “ls” la mayor cantidad de información posible sobre todos los archivos que se encuentren en su directorio home (ej: /home/guest) incluyendo aquellos que empiezan con un punto.
\end{quoting}

\begin{lstlisting}[style=Ubuntu]
~/uade_workplace/ssoo/tp1
grios@personal-pc:~$ cd /home/grios

~
grios@personal-pc:~$ ls -a

.              go          .sudo_as_admin_successful
..             .java       Templates
.azure         .lesshst    .texlive2019
.azure-shell   .local      .thunderbird
.bash_history  .mozilla    uade_workplace
.bash_logout   Music       Videos
.bashrc        .oh-my-zsh  .viminfo
.cache         Pictures    .vscode
.config        .pki        .wget-hsts
Desktop        Postman     workplace
Documents      .profile    .zcompdump
Downloads      Public      .zcompdump-grios-pc-5.8
.emacs.d       .pylint.d   .zsh_history
.gitconfig     snap        .zshrc
.gnupg         .ssh

\end{lstlisting}

\begin{quoting}
Supongamos que se desea conocer el prototipo de la función de ANSI C printf(). Tipee “man printf” y vea que sucede. ¿Es la página que estábamos buscando?
\end{quoting}

\end{subsection}

\begin{subsection}{Comando whatis}
\end{subsection}

\begin{subsection}{Comando whereis}
\end{subsection}

\begin{subsection}{Comando help}
\end{subsection}

\begin{subsection}{Comando apropos}
\end{subsection}

\begin{subsection}{Comando info}
\end{subsection}

\end{section}