\begin{section}{Usuarios}

\begin{quoting}
¿Qué es la cuenta de superusuario (root) y para qué se utiliza?
\end{quoting}

Las cuestas de superusuario o root son cuentas que poseen permisos de administrador, es decir, tienen priviligios y permisos para realizar acciones sobre el sistema. Contar con estos permisos permitirá ejecutar instrucciones sumamente utiles las cuales pueden tener un impacto positivo o negativo sobre nuestro sistema.

\hfill

\begin{quoting}
Investigue qué sucede con la cuenta de root en Ubuntu (el sistema operativo de uso en la cátedra). Investigue cómo realizar comandos a nombre del usuario root.
\end{quoting}

Como se mencionó anteriormente este comando le permite hacer cambios en el sistema, como por ejemplo, instalar una nueva aplicación con \guillemotleft sudo apt-get install your\_favorite\_app\ \guillemotright. Para ejecutar instrucciones a nombre de usuario root se puede hacer de dos formas:

\begin{itemize}

\item \textbf{su:}
comportamiento default de este comando permite \textbf{cambiar el usuario actual por un usuario root dentro de la misma sesión dentro de la terminal} y así ejecutar cuantas instrucciones necesitemos.

\begin{lstlisting}[style=Ubuntu]
~/uade_workplace/ssoo/tp1
grios@personal-pc:~$ su
Password: 

root@personal-pc:~# exit

root@personal-pc:~$ 
\end{lstlisting}

\item \textbf{sudo:}
El comando sudo tiene un comportamiento similar al anterior, pero nos permite ejecutar instrucciones sin necesidad de iniciar o cambiar de usuario.

\begin{lstlisting}[style=Ubuntu]
~/uade_workplace/ssoo/tp1
grios@personal-pc:~$ sudo apt-get install {your-favorite-package}        
[sudo] password for grios: 
\end{lstlisting}

\end{itemize}


\begin{quoting}
Investigue los comandos:
\end{quoting}


\begin{itemize}
\item \textbf{su:}
Como se mencionó anteriormente este comando permite loggearse, dentro de una sesión de terminal, con un usuario con máximos privilegios.

\begin{lstlisting}[style=Ubuntu]
~/uade_workplace/ssoo/tp1
grios@personal-pc:~$ su
Password: 
root@personal-pc:~#
\end{lstlisting}

\item \textbf{login:}
Es un comando que permite iniciar sesión, dentro del terminal, como otro usuario (puede ser root o no). Para ejecutar este comando es necesarios ejecutarlo con privilegios de root.

La forma analoga a ejecutar el comando su sería:
\begin{lstlisting}[style=Ubuntu]
~/uade_workplace/ssoo/tp1
grios@personal-pc:~$ sudo login
[sudo] password for grios: 
grios-pc login: root
Password: 
Welcome to Ubuntu 20.04 LTS (GNU/Linux 5.4.0-42-generic x86_64)

 * Documentation:  https://help.ubuntu.com
 * Management:     https://landscape.canonical.com
 * Support:        https://ubuntu.com/advantage


237 updates can be installed immediately.
0 of these updates are security updates.
To see these additional updates run: apt list --upgradable

Failed to connect to https://changelogs.ubuntu.com/meta-release-lts. Check your Internet connection or proxy settings

Your Hardware Enablement Stack (HWE) is supported until April 2025.
Last login: dom ago 23 00:56:08 -03 2020 on pts/0

root@personal-pc:~# 
\end{lstlisting}
\end{itemize}


\begin{quoting}
Investigue qué hace el comando:
\end{quoting}

\begin{itemize}

\item \textbf{adduser:}
	Nos permite crear cuentas de usuario. Durante el alta de usuario tenemos la posibilidad de configurar un directorio home y asignar un interprete de comandos, por lo general es /bin/bash. Una vez creado el usuario nos podemos loggear y configurar una contraseña

\item \textbf{addgroup:}
	Nos permite crear un grupo indicando el nombre por parámetro.
\end{itemize}

Ambos comandos necesitan privilegios de superusuario para ser ejecutados.


\begin{quoting}
Cree un nuevo usuario, cree un nuevo grupo, y agregue el usuario a ese grupo.
\end{quoting}


\begin{lstlisting}[style=Ubuntu]
~/uade_workplace/ssoo/tp1
grios@personal-pc:~$ su
Password: 

root@personal-pc:~# useradd test_user
root@personal-pc:~# groupadd test_group
root@personal-pc:~# usermod -a -G test_group test_user

root@personal-pc:~#  cat /etc/passwd | grep 'test_user'
test_user:x:1001:1001::/home/test_user:/bin/sh

root@personal-pc:~# cat /etc/group | grep 'test_user'
test_user:x:1001:
test_group:x:1002:test_user

root@personal-pc:~# exit
\end{lstlisting}

La instrucción \textbf{usermod} permite modificar la información de las cuentas de usuario preexistentes, el flag \textbf{-a -G} permite agregar al usuario a un grupo (append group)

\begin{quoting}
Investigue qué hace el comando:
\end{quoting}

\begin{itemize}
\item \textbf{deluser:}
Este comando permite quitar usuarios de un grupo.
\begin{lstlisting}[style=Ubuntu]
~/uade_workplace/ssoo/tp1
grios@personal-pc:~$ sudo deluser {your_favorite_user} {your_favorite_group} 
\end{lstlisting}

\item \textbf{delgroup:}
Este comando permite eliminar un grupo.
\begin{lstlisting}[style=Ubuntu]
grios@personal-pc:~$ sudo delgroup {your_favorite_group} 
\end{lstlisting}
\end{itemize}

\begin{quoting}
Borre el usuario creado anteriormente (incluyendo el borrado de su directorio en home y todos sus
archivos).
\end{quoting}

\begin{lstlisting}[style=Ubuntu]
~/uade_workplace/ssoo/tp1
grios@personal-pc:~$ sudo deluser test_user --remove-home  
[sudo] password for grios: 
Looking for files to backup/remove ...
Removing user `test_user' ...
Warning: group `test_user' has no more members.
Done.

\end{lstlisting}


\begin{quoting}
Investigue cómo hacer para saber todos los grupos a los que pertenece un usuario.
\end{quoting}

Para lograr esto encontramos 3 formas:

\begin{itemize}

\item \textbf{Viendo files del sistema:}
En el archivo /etc/group se puede encontrar información relevante de todos los grupos creados en el sistema, y si usamos el comando grep podemos obtener información del grupo que nos interese, como por ejemplo los usuarios que pertenecen a él.

\begin{lstlisting}[style=Ubuntu]
~/uade_workplace/ssoo/tp1
grios@personal-pc:~$ cat /etc/group | grep 'test_user'
test_user:x:1001:
test_group:x:1002:test_user
\end{lstlisting}

\item \textbf{Comando getent:}
El comando getent nos listará todos los grupos existentes en nuestra máquina, si usamos la misma tactica que antes podemo hacer un grep y ver a qué grupo pertenece un usuario puntual.

\begin{lstlisting}[style=Ubuntu]
~/uade_workplace/ssoo/tp1
grios@personal-pc:~$ getent group | grep 'test_user'
test_group:x:1002:test_user
test_user:x:1001:
\end{lstlisting}

\item \textbf{Comando groups:}
El comando groups quizá es el más adecuado para averiguar todos los grupos a los que pertenece un usuario, la forma de ejecutarlo es:

\begin{lstlisting}[style=Ubuntu]
~/uade_workplace/ssoo/tp1
grios@personal-pc:~$ groups test_user
test_user : test_user test_group
\end{lstlisting}

\end{itemize}
\end{section}