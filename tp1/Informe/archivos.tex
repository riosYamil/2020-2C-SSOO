\begin{section}{Archivos}

\begin{quoting}
¿Qué hacen los siguientes comandos?
\end{quoting}


\begin{itemize}

\item \textbf{cp:}
Permite copiar un archivo o varios desde un directorio origen a un destino.
\begin{lstlisting}[style=Ubuntu]
~uade_workplace/ssoo/tp1
grios@personal-pc:~$ cp {sourche_path} {destination_path}
\end{lstlisting}

\item \textbf{mv:}
Permite mover un archivo (o directorio) desde un directorio origen a uno de destino. Además de esta funcionalidad permite renombrar un archivo o directorio. Por ejemplo:

\begin{itemize}
\item Move directorio
\begin{lstlisting}[style=Ubuntu]
~uade_workplace/ssoo/tp1
grios@personal-pc:~$ mv folder_test /your/favorite/destination_path/
\end{lstlisting}

\item Move archivo
\begin{lstlisting}[style=Ubuntu]
~uade_workplace/ssoo/tp1
grios@personal-pc:~$ mv test_file.txt /your/favorite/destination_path/
\end{lstlisting}

\item Renombrar directorio
\begin{lstlisting}[style=Ubuntu]
~uade_workplace/ssoo/tp1
grios@personal-pc:~$ mv /your/favorite/directory/ /your/favorite/directory/ your_favorite_new_directory_name
\end{lstlisting}

\item Renombrar archivos
\begin{lstlisting}[style=Ubuntu]
~uade_workplace/ssoo/tp1
grios@personal-pc:~$ mv test_file.txt test_file_with_new_name.txt
\end{lstlisting}
\end{itemize}

\item \textbf{rm:}
Permite borrar archivos y si se usa recursivamente (flag -r) permite borrar directorios
\begin{itemize}
\item Borrar directorio
\begin{lstlisting}[style=Ubuntu]
~uade_workplace/ssoo/tp1
grios@personal-pc:~$ rm folder_test
\end{lstlisting}

\item Borrar archivo
\begin{lstlisting}[style=Ubuntu]
~uade_workplace/ssoo/tp1
grios@personal-pc:~$ mv -r /your/favorite/directory/
\end{lstlisting}

\end{itemize}


\item \textbf{scp:}
Este comando nos permite realizar transferencias de archivos o directorios desde nuestra máquina local a servidores remoto, y también permite la transferencia entre servidores remotos.

\begin{itemize}
\item Transferencia local a remoto
\begin{lstlisting}[style=Ubuntu]
~uade_workplace/ssoo/tp1
grios@personal-pc:~$ scp test_file.txt test_user_a@domain.com:/home/test_user_a
\end{lstlisting}

\item Transferencia remoto a local
\begin{lstlisting}[style=Ubuntu]
~uade_workplace/ssoo/tp1
grios@personal-pc:~$ scp test_user_a@domain.com:/home/test_user_a/test_file.txt /home/grios
\end{lstlisting}

\item Transferencia remoto a remoto
\begin{lstlisting}[style=Ubuntu]
~uade_workplace/ssoo/tp1
grios@personal-pc:~$ scp test_user_a@domain.com:/home/test_user_a/test_file.txt test_user_b@domain.com:/home/test_user_b/
\end{lstlisting}

\end{itemize}

\item \textbf{telnet:}
Este comando permite acceder de manera remota a un servidor mediante el protocolo telnet y ejecutar comandos de manera remota, hoy por hoy este comando es reemplazado en muchos casos por ssh, dado que telnet no es considerado un método seguro de transferencia de datos.
\begin{lstlisting}[style=Ubuntu]
~uade_workplace/ssoo/tp1
grios@personal-pc:~$ telnet {your_favorite_ip}:{your_favorite_port}
\end{lstlisting}

\item \textbf{ssh:}
Este comando tiene las mismas prestaciones que el comando telnet, pero con la difernecia que se usa un protocolo ssh estableciendo un canal seguro ya que las información se transporta encriptada
\begin{lstlisting}[style=Ubuntu]
~uade_workplace/ssoo/tp1
grios@personal-pc:~$ ssh test_user@yourfavoriteserver.domain.com {your_favorite_command}
\end{lstlisting}

\item \textbf{touch:}
Este comando generalmente se usa para crear archivos vacíos o cambiar las tiempos de actualización de un archivo preexistente (sólo el tiempo de acceso y tiempo de modificación).
\begin{lstlisting}[style=Ubuntu]
~uade_workplace/ssoo/tp1
grios@personal-pc:~$ touch test_file.txt
\end{lstlisting}
\end{itemize}


\begin{quoting}
A la hora de referirse a archivos, se puede usar tanto su dirección relativa (al directorio en el que se
encuentra situado) o absoluta. Sitúese como root dentro del directorio /root. Luego copie el archivo .bashrc
a la ruta absoluta /var/.bashrc. Ahora, mueva ese archivo desde esa dirección hasta /home/.bashrc sin
desplazarse del directorio inicial (/root).
\end{quoting}

\begin{lstlisting}[style=Ubuntu]
~uade_workplace/ssoo/tp1
grios@personal-pc:~$ su
Password: 
root@personal-pc:/home/grios# cd /root/
root@personal-pc:~# cp .bashrc /var/.bashrc
root@personal-pc:~# cp /var/.bashrc /home/.bashrc
root@personal-pc:~# exit

~uade_workplace/ssoo/tp1
grios@personal-pc~$ l /var | grep 'bashrc' 
-rw-r--r--  1 root root     3,1K ago 23 12:55 .bashrc

~uade_workplace/ssoo/tp1
grios@personal-pc~$ l /home | grep 'bashrc' 
-rw-r--r--  1 root  root  3,1K ago 23 12:56 .bashrc
                                                             
\end{lstlisting}

\end{section}